\documentclass{article}
\usepackage[utf8]{inputenc}
\usepackage[spanish]{babel}
\usepackage{graphicx}
\usepackage{anysize}
\usepackage{fancyhdr} 
\usepackage[export]{adjustbox}
\usepackage{titlesec}
\usepackage{enumitem}
\usepackage{amsmath}
\usepackage{amssymb}
\usepackage{listings}
\usepackage{xcolor}
% \usepackage{hyperref}
% \usepackage{float}
% \usepackage{tabu}

% Izquierda, derecha, arriba, abajo
\marginsize{1.5cm}{2cm}{1cm}{1cm} 
\renewcommand{\familydefault}{\sfdefault}
\decimalpoint%

\setlength{\parindent}{0in}
\titleformat*{\section}{\large\bfseries}

% Para insert código
\definecolor{codegreen}{rgb}{0,0.6,0}
\definecolor{codegray}{rgb}{0.5,0.5,0.5}
\definecolor{codepurple}{rgb}{0.58,0,0.82}
\definecolor{backcolour}{rgb}{1,1,1}

\usepackage{textcomp}
\lstset{upquote=true}
\lstdefinestyle{mystyle}{
    backgroundcolor=\color{backcolour},   
    commentstyle=\color{codegreen},
    keywordstyle=\color{magenta},
    numberstyle=\tiny\color{codegray},
    stringstyle=\color{codepurple},
    basicstyle=\ttfamily\footnotesize,
    breakatwhitespace=false,         
    breaklines=true,                 
    captionpos=b,                    
    keepspaces=true,                 
    % numbers=left,                    
    % numbersep=5pt,                  
    showspaces=false,                
    showstringspaces=false,
    showtabs=false,                  
    tabsize=2
}

\lstset{style=mystyle}

\newcommand{\materia}{BDD}
\newcommand{\clave}{2947}
\newcommand{\profesor}{Ing. Rodriguez Campos \textsc{Jorge Alberto}}
\newcommand{\grupo}{1}
\newcommand{\semestre}{2021-1}

\newcommand{\alumno}{
    Francisco Pablo \textsc{Rodrigo}  \\ 
    Flores Martinez \textsc{Emanuel}   
}

\newcommand{\actividad}{Práctica 08}
\newcommand{\titulo}{Transparencia de distribución para la instrucciones DML}

\newcommand{\fechaEntrega}{11 de enero de 2021}

\newcommand{\codedir}{bdd_prac_08-codigo}
\graphicspath{{assets/}{bdd_prac_08.assets/}}

%%%%%%%%%%%%%%%%%%%% ENCABEZADO %%%%%%%%%%%%%%%%%%%%%%%%%%%%
\pagestyle{fancy}
\fancyhf{}
\renewcommand{\headrulewidth}{0pt}
\fancyhead[R]{% Left header
    \begin{tabular}{l}
        \materia \\ 
        \actividad%
    \end{tabular}
    \,% Space
    \rule[-1.75\baselineskip]{0pt}{0pt}
    % Strut to ensure a 1/4 \baselineskip between image and header rule
    \includegraphics[height=3\baselineskip,valign=c]{unam}
}
\setlength{\headsep}{0.2in}

\begin{document}
%%%%%%%%%%%%%%%%%%% DATOS PORTADA %%%%%%%%%%%%%%%%%%%%%%%%
\thispagestyle{empty}
\begin{minipage}[t][5cm][t]{0.2\linewidth}
    \includegraphics[width=2.5cm]{unam.jpg}
    \vspace{10cm}

    \includegraphics[width=2.5cm]{fiblack}
\end{minipage}
\begin{minipage}[t]{0.7\linewidth}
    \vspace{-2.5cm}
    \LARGE{\textbf{Universidad Nacional Autónoma de México}}\\
    \Large{\textbf{Facultad de Ingeniería}} \\

    \large{\semestre}\\[2cm]

    \large{\textbf{\materia (\clave)}}\\
    \large{\textbf{Gpo: 1}}\\[5mm]
    \large{\textbf{Profesor:} \profesor}\\ [1.5cm]
    \begin{center}
        \LARGE{\textbf{\actividad}}\\
        \LARGE{\textbf{\titulo}}\\
    \end{center}

    \vspace{3.3cm}

    % \large{\textbf{Alumno:} \alumno} \\[1.5cm]
    \large{
        \begin{itemize}[
            noitemsep,
            % topsep=0pt,
            % parsep=0pt,
            % partopsep=0pt,
            % labelwidth=5cm,
            align=left,
            % itemindent=2cm
        ]
            \item [\textbf{Alumno(s):}] 
            \begin{flushright}
                \alumno
            \end{flushright}
        \end{itemize}
    } \vspace{1.5cm}

    \begin{flushright}
        \fechaEntrega%
    \end{flushright}
\end{minipage}

\newpage
%%%%%%%%%%%%%%%%%%% CONTENIDO %%%%%%%%%%%%%%%%%%%%%%%%

\section*{Introducción}

A manera de recordatorio, en las prácticas pasadas se realizó la transparencia
de distribución para instrucciones SELECT. En esta práctica se implementará la
transparencia de distribución para instrucciones DML. En manejador de Oracle no
ofrece mecanismos especializados para tratar las instrucciones DML en un
ambiente distribuido, por lo cual, corresponde al responsable del software
encargarse de implementarlas de manera manual. Dicho de otro modo, es imposible
que el manejador ofrezca dicha transparencia debido a que cada caso de estudio
tiene reglas establecidas con base en la arquitectura y licencias que se cuenten
o que se puedan aquirir.\\

La transparencia de localización se debe implementar utilizando programación
SQL, PL/SQL para el caso de Oracle. Se requiere tener conocimientos sólidos
sobre el funcionamiento de los \texttt{triggers}, procedimientos alamacenados y
funciones. Al realizar la práctica se deben de identificar las estrategias que
permitan optimizar el uso de la red. Por ellos se recomienda implementar por
separado los triggers del sitio 1 y del sitio 2.


\section*{Objetivos}

\begin{itemize}
  \item Para todas las actividades de esta práctica, considerar el esquema de
  fragmentación de la práctica anterior.  
  \item La práctica puede realizarse en equipo de 2 personas. Para este caso
    cada integrante deberá realizar todas las actividades en sus respectivos
ambientes.
\end{itemize}

\section*{C1. Extracto del trigger para \texttt{pais} que implementa la 
sentencia \texttt{update}}
\textbf{Flores Martinez Emanuel}
\begin{itemize}
  \item \texttt{s-03-efm-pais-trigger.sql}
  \lstinputlisting[language=SQL,firstline=29,lastline=70]
  {\codedir/s-03-efm-pais-trigger.sql}
\end{itemize}

\section*{C2. Extracto del trigger para \texttt{suscriptor} que implementa la 
sentencia \texttt{insert}}
\textbf{Francisco Pablo Rodrigo}
\begin{itemize}
  \item \texttt{s-03-rfp-suscriptor-n1-trigger.sql}
  \lstinputlisting[language=SQL,firstline=12,lastline=66]
  {\codedir/s-03-rfp-suscriptor-n1-trigger.sql}

  \item \texttt{s-03-rfp-suscriptor-n2-trigger.sql}
  \lstinputlisting[language=SQL,firstline=12,lastline=68]
  {\codedir/s-03-rfp-suscriptor-n2-trigger.sql}
\end{itemize}

\section*{C3. Extracto del trigger para \texttt{pago\_suscriptor} que 
implementa la sentencia \texttt{insert}}

\textbf{Flores Martinez Emanuel}
\begin{itemize}
  \item \texttt{s-03-efm-pago-suscriptor-n1-trigger.sql}
  \lstinputlisting[language=SQL,firstline=11,lastline=38]
  {\codedir/s-03-efm-pago-suscriptor-n1-trigger.sql}

  \item \texttt{s-03-efm-pago-suscriptor-n2-trigger.sql}
  \lstinputlisting[language=SQL,firstline=11,lastline=39]
  {\codedir/s-03-efm-pago-suscriptor-n2-trigger.sql}
\end{itemize}

\section*{C4. Extracto del trigger para \texttt{articulo} que 
implementa la sentencia \texttt{insert}}

\textbf{Francisco Pablo Rodrigo}
\begin{itemize}
  \item \texttt{s-03-rfp-articulo-n1-trigger.sql}
  \lstinputlisting[language=SQL,firstline=14,lastline=27]
  {\codedir/s-03-rfp-articulo-n1-trigger.sql}

  \item \texttt{s-03-rfp-articulo-n2-trigger.sql}
  \lstinputlisting[language=SQL,firstline=14,lastline=29]
  {\codedir/s-03-rfp-articulo-n2-trigger.sql}
\end{itemize}

\section*{C5. Salida del Validador.}

\begin{itemize}
  \item \textbf{Salida de Francisco Pablo Rodrigo}
    \lstinputlisting[language=SQL,firstline=51,lastline=106,
    basicstyle=\scriptsize] {\codedir/validador.txt}
    \lstinputlisting[language=SQL,firstline=129,basicstyle=\scriptsize]
    {\codedir/validador.txt}

  \item \textbf{Salida de Flores Martínez Emanuel}
    \lstinputlisting[language=SQL,firstline=51,lastline=106,
    basicstyle=\scriptsize] {\codedir/validador-efm.txt}
    \lstinputlisting[language=SQL,firstline=129,basicstyle=\scriptsize]
    {\codedir/validador-efm.txt}
\end{itemize}

\section*{Comentarios y conclusiones}

En esta práctica aprendimos cómo utilizar \texttt{Instead of DML Triggers},
un ente de la base de datos Oracle que nos permitió atrapar inserciones,
actualizaciones y eliminaciones de las vistas globales y redireccionar su
ejecución al fragmento del sitio correspondiente.\\

Así mismo, reconocimos la importancia de la programación como parte del
desarrollo de software distribuido ya que no hay una herramienta que realice la
distribución de nuestra base de datos y proporcione transparencia de
distribución. Es importante resaltar la importancia de las buenas prácticas y
del código comentado, de lo contrario es fácil perderse a la hora de realizar
los triggers que van a servir para mantener nuestros datos consistentes.

\renewcommand\refname{Bibliografía}
\begin{thebibliography}{99}
    \bibitem{oracle} Oracle. \textit{Oracle Database Documentation} en 
        \texttt{https://docs.oracle.com/en/database/oracle/\\oracle-database/%
        index.html}
    \bibitem{oracledb} Oracle Database. \textit{Database PL/SQL Language 
        Reference} en 
        \texttt{https://docs.oracle.com/en/database/oracle/\\
        oracle-database/12.2/lnpls/database-pl-sql-language-reference.pdf}
\end{thebibliography}

\end{document}
