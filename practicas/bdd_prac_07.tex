\documentclass{article}
\usepackage[utf8]{inputenc}
\usepackage[spanish]{babel}
\usepackage{graphicx}
\usepackage{anysize}
\usepackage{fancyhdr} 
\usepackage[export]{adjustbox}
\usepackage{titlesec}
\usepackage{enumitem}
\usepackage{amsmath}
\usepackage{amssymb}
\usepackage{listings}
\usepackage{xcolor}
% \usepackage{hyperref}
% \usepackage{float}
% \usepackage{tabu}

% Izquierda, derecha, arriba, abajo
\marginsize{1.5cm}{2cm}{1cm}{1cm} 
\renewcommand{\familydefault}{\sfdefault}
\decimalpoint%

\setlength{\parindent}{0in}
\titleformat*{\section}{\large\bfseries}

% Para insert código
\definecolor{codegreen}{rgb}{0,0.6,0}
\definecolor{codegray}{rgb}{0.5,0.5,0.5}
\definecolor{codepurple}{rgb}{0.58,0,0.82}
\definecolor{backcolour}{rgb}{1,1,1}

\usepackage{textcomp}
\lstset{upquote=true}
\lstdefinestyle{mystyle}{
    backgroundcolor=\color{backcolour},   
    commentstyle=\color{codegreen},
    keywordstyle=\color{magenta},
    numberstyle=\tiny\color{codegray},
    stringstyle=\color{codepurple},
    basicstyle=\ttfamily\footnotesize,
    breakatwhitespace=false,         
    breaklines=true,                 
    captionpos=b,                    
    keepspaces=true,                 
    % numbers=left,                    
    % numbersep=5pt,                  
    showspaces=false,                
    showstringspaces=false,
    showtabs=false,                  
    tabsize=2
}

\lstset{style=mystyle}

\newcommand{\materia}{BDD}
\newcommand{\clave}{2947}
\newcommand{\profesor}{Ing. Rodriguez Campos \textsc{Jorge Alberto}}
\newcommand{\grupo}{1}
\newcommand{\semestre}{2021-1}

\newcommand{\alumno}{
    Francisco Pablo \textsc{Rodrigo}  \\ 
    Flores Martinez \textsc{Emanuel}   
}

\newcommand{\actividad}{Práctica 07}
\newcommand{\titulo}{Transparencia de distribución para la instrucción SELECT}

\newcommand{\fechaEntrega}{07 de enero de 2021}

\newcommand{\codedir}{bdd_prac_07-codigo}
\graphicspath{{assets/}{bdd_prac_07.assets/}}

%%%%%%%%%%%%%%%%%%%% ENCABEZADO %%%%%%%%%%%%%%%%%%%%%%%%%%%%
\pagestyle{fancy}
\fancyhf{}
\renewcommand{\headrulewidth}{0pt}
\fancyhead[R]{% Left header
    \begin{tabular}{l}
        \materia \\ 
        \actividad%
    \end{tabular}
    \,% Space
    \rule[-1.75\baselineskip]{0pt}{0pt}
    % Strut to ensure a 1/4 \baselineskip between image and header rule
    \includegraphics[height=3\baselineskip,valign=c]{unam}
}
\setlength{\headsep}{0.2in}

\begin{document}
%%%%%%%%%%%%%%%%%%% DATOS PORTADA %%%%%%%%%%%%%%%%%%%%%%%%
\thispagestyle{empty}
\begin{minipage}[t][5cm][t]{0.2\linewidth}
    \includegraphics[width=2.5cm]{unam.jpg}
    \vspace{10cm}

    \includegraphics[width=2.5cm]{fiblack}
\end{minipage}
\begin{minipage}[t]{0.7\linewidth}
    \vspace{-2.5cm}
    \LARGE{\textbf{Universidad Nacional Autónoma de México}}\\
    \Large{\textbf{Facultad de Ingeniería}} \\

    \large{\semestre}\\[2cm]

    \large{\textbf{\materia (\clave)}}\\
    \large{\textbf{Gpo: 1}}\\[5mm]
    \large{\textbf{Profesor:} \profesor}\\ [1.5cm]
    \begin{center}
        \LARGE{\textbf{\actividad}}\\
        \LARGE{\textbf{\titulo}}\\
    \end{center}

    \vspace{3.3cm}

    % \large{\textbf{Alumno:} \alumno} \\[1.5cm]
    \large{
        \begin{itemize}[
            noitemsep,
            % topsep=0pt,
            % parsep=0pt,
            % partopsep=0pt,
            % labelwidth=5cm,
            align=left,
            % itemindent=2cm
        ]
            \item [\textbf{Alumno(s):}] 
            \begin{flushright}
                \alumno
            \end{flushright}
        \end{itemize}
    } \vspace{1.5cm}

    \begin{flushright}
        \fechaEntrega%
    \end{flushright}
\end{minipage}

\newpage
%%%%%%%%%%%%%%%%%%% CONTENIDO %%%%%%%%%%%%%%%%%%%%%%%%

\section*{Introducción}

En prácticas pasadas se implementó la transparencia de mapeos locales por 
medio de un artefacto de Oracle llamado `ligas', en esta práctica se 
implementará la transparencia de localización y fragmentación por medio de
sinónimos y vistas. Además, se explicará como realizar el manejo de archivos
blob de manera destribuida para lo cual se requiere tener fuertes conocimientos
de PL/SQL ya que se empleará por medio de funciones, procedimientos 
almacenados, cursores, etc.\\

A nivel general existen 2 estrategias para implementar el uso de archivo LOB
en un sistema de base de datos distribuido. En la primera estrategia se 
recolectarán todos los datos de la tabla lo cual puede ser costoso en tiempo
y recursos, empero esto dependerá de la consulta ejectuda del lado del cliente
final. La segunda estrategía esta más optimizada para cuando se requiera
hacer uso de LOB en particular. De hecho usarlo para consultar todos los 
registros de la tabla podría ser contraproducente.

\section*{Objetivos}

Comprender e implementar transparencia de localización y de fragmentación 
aplicados a una base de datos distribuida para realizar operaciones
\texttt{select}.

\section*{C1. Código SQL que contiene la definición de una vista que requiere 
  el manejo de datos BLOB con estrategia 1 y 2.}

  \lstinputlisting[language=SQL,firstline=126,lastline=148]
  {\codedir/s-05-rfpbd_s2-soporte-blobs.sql}

\section*{C2. Código SQL para una consulta empleando las vistas con 
  fragmentación híbrida }

  Nota: La transparencia de fragmentación implica `esconder' el uso de 
  fragmentos, por esa razón para hacer el conteo de datos en la tabla
  suscriptor (fragmentación híbrida) simplemente escribimos \texttt{count(*)} 
  y lo aplicamos a la vista \texttt{suscriptor}. \\

  En los otros nivel de transparencia se observa que antes de hacer el conteo
  de registros era necesario primero hacer la reconstrucción de la tabla
  por medio de uniones y joins.

  \lstinputlisting[language=SQL,firstline=10,lastline=16]
  {\codedir/s-06-rfp-consultas-fragmentacion.sql}

%\newpage
\section*{C3. Salida de ejecución del script de validación}

\begin{itemize}
  \item  \textbf{Salida de Flores Martinez Emanuel}
  \lstinputlisting[firstline=47,lastline=98]
    {bdd_prac_07.assets/efm-validador.txt}
  \lstinputlisting[firstline=126]
    {bdd_prac_07.assets/efm-validador.txt}

  \item \textbf{Salida de Francisco Pablo Rodrigo}
  \lstinputlisting[firstline=47,lastline=98]
    {bdd_prac_07.assets/rfp-validador.txt}
  \lstinputlisting[firstline=126]
    {bdd_prac_07.assets/rfp-validador.txt}
\end{itemize}

\section*{Comentarios y conclusiones}

La transparencia de fragmentación nos ofrece sencillez del lado del usuario 
para realizar consultas como si estuvieramos de forma centralizada. Sin 
embargo, siempre se debe tomar en cuenta que al estar en un ambiente 
distribuido los costos de algunas operaciones se incrementarán y más si 
trabajamos con archivos LOB. Por lo cual a pesar de tener implementado el 
nivel de transparencia de mayor jerárquica debemos seguir afinando nuestro
proyecto para que funcione de la manera más óptima posible.\\

Por otra parte, implementar LOBs de manera distruida representa un reto y 
para alcanzar dicho objetivo se tienen que tener conocimientos sólidos de 
como funciona el manejador en donde se este implementando. De hecho, en esta
práctica aprendimos a utilizar los objetivos tipo que van de la mano con las 
funciones.Y a su vez aprendimos a realizar transacciones en una función y como
evitar que estas afecten la integridad de la base de datos.

\renewcommand\refname{Bibliografía}
\begin{thebibliography}{99}
    \bibitem{oracle} Oracle. \textit{Oracle Database Documentation} en 
        \texttt{https://docs.oracle.com/en/database/oracle/\\oracle-database/%
        index.html}
    \bibitem{oracledb} Oracle Database. \textit{Database PL/SQL Language 
        Reference} en 
        \texttt{https://docs.oracle.com/en/database/oracle/\\
        oracle-database/12.2/lnpls/database-pl-sql-language-reference.pdf}
\end{thebibliography}

\end{document}
